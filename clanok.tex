% Metódy inžinierskej práce

\documentclass[10pt,oneside,slovak,a4paper]{article}

\usepackage[slovak]{babel}
%\usepackage[T1]{fontenc}
\usepackage[IL2]{fontenc} % lepšia sadzba písmena Ľ než v T1
\usepackage[utf8]{inputenc}
\usepackage{graphicx}
\usepackage{url} % príkaz \url na formátovanie URL
\usepackage{hyperref} % odkazy v texte budú aktívne (pri niektorých triedach dokumentov spôsobuje posun textu)

\usepackage{cite}
%\usepackage{times}

\pagestyle{headings}

\title{Využitie gamifikácie vo výučbe cudzích jazykov\thanks{Semestrálny projekt v predmete Metódy inžinierskej práce, ak. rok 2020/21, vedenie: Jozef Sitarčík}} % meno a priezvisko vyučujúceho na cvičeniach

\author{Peter Janoš\\[2pt]
	{\small Slovenská technická univerzita v Bratislave}\\
	{\small Fakulta informatiky a informačných technológií}\\
	{\small \texttt{xjanosp@stuba.sk}}
	}

\date{\small 20.8.2020} % upravte



\begin{document}

\maketitle

\begin{abstract}
Ovládanie cudzieho jazyka sa v dnešnej dobe považuje za nevyhnutné. Mnoho ľudí sa s cudzím jazykom, ako napríklad angličtina, stretáva už v ranom veku v školách a potom s ňou prichádzajú do kontaktu takmer každý deň na internete, v hudbe, vo filmoch alebo v článkoch, čo im pomáha sa ďalej rozvíjať. Existuje ale aj skupina ľudí, ktorí takéto možnosti nemali a chcú sa cudzí jazyk naučiť sami. No učenie sa cudzieho jazyka je zložitý proces a preto vznikli nové spôsoby vzdelávania, ktoré ho má zjednodušiť. Jeden z týchto nových spôsobov je takzvaná gamifikácia. Gamifikácia je relatívne nový koncept, ktorý spája zábavu s učením. Využíva herné elementy a dizajn, čo v teórii spríjemňuje atmosféru pri učení a taktiež odmieňa učiaceho sa za jeho pokroky, čo ho motivuje sa naďalej zlepšovať. Výsledky štúdií, ktoré sa venovali efektivite tejto stratégie ale preukázali zmiešané výsledky. V tomto článku sa budeme bližšie venovať gamifikácií ako takej, učeniu sa cudzieho jazyka a aj tomu ako môže táto stratégia pomáhať motivovať účastníka zapájať sa do tohto vzdelávacieho procesu.
\end{abstract}



\section{Úvod} \label{uvod}

Počítačové a mobilné hry sa stali súčasťou života mnohých ľudí. Aj preto môžeme čoraz častejšie vidieť, že sa nejakú formu hier snažia učitelia implementovať aj do vzdelávania. Takýmto spôsobom učenia sa je aj stále viac a viac populárnejšia gamifikácia, ktorej sa budeme venovať v tomto článku. Gamifikácia, by mohla byť novou metódou, ktorá by umožňovala učiteľom viac motivovať svojich študentov, aby sa efektívnejšie a rýchlejšie naučili druhý jazyk. 

Keďže je gamifikácia stále relatívne novým konceptom, nebolo na ňu vykonaných veľa štúdií, ktoré by potvrdzovali alebo vyvrátili jej efektívnosť. Štúdie ktoré boli vykonané poukazovali na pozitívne výsledky pri vplyve na motiváciu učiaceho sa, no niektoré zase poukazovali na presný opak. Gamifikácia zatiaľ našla svoje využitie hlavne v marketingových alebo obchodných oblastiach. ~\cite{garland2015gamification}



\section{Dôležité definicie} \label{definicie}
\subsection{Definícia gamifikácue} \label{gamifikacia}
V tejto časti sa budeme venovať gamifikácií ako takej. Čo je to vlastne gamifikácia? Aj keď sa na prvý pohľad môže zdať, že gamifikácia a hra je to isté, nie je tomu tak. Hry ako počítačové alebo stolné hry hráme pre zábavu. Gamifikácie je však metóda, ktorú môžme použiť pre zlepšenie používateľského zážitku pri práci, alebo učení sa. Využíva herné prvky a herný dizajn v nehernom kontexte, teda napríklad za iným účelom ako je samotné hranie hry. Ako napríklad získavanie bodov pri úspešnom riešení problému, odomykane nových úrovní alebo aj odmenenie virtuálnym odznakom. Princíp gamifikácie môžeme vidieť v aplikáciách ako napríklad:

\begin{itemize}
    \item Duolingo - Aplikácia na učenie cudzích jazykov, ktorá používateľa odmieňa za pokroky. Núti používateľa neustále opakovať nové učivo.
    \item Waze - Navigačná aplikácia, ktorá využíva gamifikáciu na získavanie informácií, čím zlepšuje svoju presnosť.
    \item Adidas - Internetový obchod, ktorý využíva gamifikáciu pri odmieňaní používateľa bodmi pri každom nákupe. Čím viac bodov má, tým má väčšie benefity.
\end{itemize}



\subsection{Motivácia a učenie} \label{motivacia}
Motivácia hrá pri učení sa veľmi dôležitú rolu. Byť motivovaný znaméná byť odhodlaný začať na niečom pracovať a aj to dokončiť. Človeka, ktorý necíti chuť ani inšpiráciu pracovať na niečom môžeme nazvať nemotivovaným. Na druhej strane, človeka ktorý je podnietený a pripravený dotiahnuť svoju prácu do konca, môžeme nazvať motivovaným. Motivácia sa u ľudí nelíši len úrovňou (ako veľmi sú motivovaní), ale aj typom motivácie. Typy motivácie rozoznávame podľa postojov a cieľov, ktoré vedú človeka k tomu, aby konal. Tieto typy rozdeľujeme na vnútornú a vonkajšiu motiváciu. ~\cite{ryan2000intrinsic}

\subsubsection{Vnútorná motivácia} \label{vnutorna}
Vnútorná motivácia je definovaná ako vykonávanie činnosti skôr pre uspokojenie a zábavu, ktorú daná činnosť človeku prináša, ako pre nejakú odmenu alebo ako dôsledok nátlaku. Dobrým príkladom vnútornej motivácie je napríklad študent, ktorý je veľmi motivovaný učiť sa a pracovať na sebe, len kvôli tomu, že chce získať nové vedomosti a zručnosti a nie len kvôli tomu, aby dostal dobrú známku. ~\cite{ryan2000intrinsic}

\subsubsection{Vonkajšia motivácia} \label{vonkajsia}
Externá motivácia je presný opak. Je to vykonávanie nejakej činnosti len pre dosiahnutie nejakého výsledku. Ako napríklad odmena alebo dobrá známka. Alebo naopak, aby sa vyhol nejakému trestu. Ako príklad môžme použiť študenta, ktorý na sebe pracuje len kvôli strachu z môžného trestu od rodičov.
~\cite{ryan2000intrinsic}
Lepper (1988) tvrdí, že vnútorná motivácia poukazuje na lepšie výsledky. Výsledkom vonkajšej motivácie je nižšia úroveň motivácie a spomalenie procesu učenia sa po uplynutí nejakého času a to aj v gamifikovanom kontexte. To však neznamená, že vonkajšia motivácia nemôže byť používaná vo výučbe. Prvky vonkajšej motivácie môžu byť implementované do aktivít a môžu byť použité spôsobom, ktorý posilňuje vnútornú motiváciu študentov. Čo znamená, že v závislosti od toho ako je gamifikácia použitá a implementovaná, dokáže posilniť vonkajšiu ale aj vnútornú motiváciu študenta.
~\cite{garland2015gamification}



%\acknowledgement{Ak niekomu chcete poďakovať\ldots}


% týmto sa generuje zoznam literatúry z obsahu súboru literatura.bib podľa toho, na čo sa v článku odkazujete
\bibliography{literatura}
\bibliographystyle{plain}
\end{document}
